%%%%%%%%%%%%%%%%%%%%%%%%%%%%%%%%%%%%%%%%%
% Medium Length Professional CV
% LaTeX Template
% Version 2.0 (8/5/13)
%
% This template has been downloaded from:
% http://www.LaTeXTemplates.com
%
% Original author:
% Rishi Shah
%
% Important note:
% This template requires the resume.cls file to be in the same directory as the
% .tex file. The resume.cls file provides the resume style used for structuring the
% document.
%
%%%%%%%%%%%%%%%%%%%%%%%%%%%%%%%%%%%%%%%%%

%----------------------------------------------------------------------------------------
%	PACKAGES AND OTHER DOCUMENT CONFIGURATIONS
%----------------------------------------------------------------------------------------

\documentclass{resume} % Use the custom resume.cls style

\usepackage[left=0.75in,top=0.6in,right=0.75in,bottom=0.6in]{geometry}
% Document margins

\usepackage{comment}
\usepackage{hyperref}
\usepackage{enumerate}
\newcommand{\tab}[1]{\hspace{.2667\textwidth}\rlap{#1}}
\newcommand{\itab}[1]{\hspace{0em}\rlap{#1}}
\newenvironment{cvEnum}
{ \begin{enumerate}[]
    \setlength{\itemsep}{0pt}
    \setlength{\parskip}{0pt}
    \setlength{\parsep}{0pt}     }
{ \end{enumerate}                  }
\name{Zeynep Hakguder} % Your name
\address{539 N 24th St Apt 19, Lincoln, NE 68503} % Your address
%\address{123 Pleasant Lane \\ City, State 12345} % Your secondary addess (optional)
\address{+1 (402) 853-9069 \\ \href{mailto:zhakguder@cse.unl.edu}{zphakguder@gmail.com}} % Your phone number and email

\begin{document}

%----------------------------------------------------------------------------------------
%	EDUCATION SECTION
%----------------------------------------------------------------------------------------

\begin{rSection}{Education}

{\bf University of Nebraska-Lincoln} \hfill {\em August 2017 - May
  2020 (Expected)}
\\ PhD in Computer Science, \it{Machine Learning Specialization} \hfill { Cumulative GPA: 3.969 }

\end{rSection}

\begin{rSection}{Experience}
{\bf Teaching Assistant}
\begin{cvEnum}
\item Design and Analysis of Algorithms  \hfill {\em Fall 2018 ---}
\item Introduction to Machine Learning \hfill {\em Summer 2018}
\item Data Structures and Algorithms \hfill {\em Summer 2017 \& Spring 2018}
\item Introduction to Python Programming \hfill {\em Fall 2017}
\end{cvEnum}
{\bf Research Assistant} \hfill {\em January 2017-May 2018}
\begin{cvEnum}
\item SBBI Lab, Department of Computer Science and Engineering
    \end{cvEnum}

\end{rSection}



%----------------------------------------------------------------------------------------
%	TECHNICAL STRENGTHS SECTION
%----------------------------------------------------------------------------------------

\begin{rSection}{SKILLS}

\begin{tabular}{ @{} >{\bfseries}l @{\hspace{6ex}} l }
Programming Languages \ & Python, JavaScript \\
Scripting Languages \ & Bash, AWK, sed, \LaTeX, SQL \\
Deep Learning Libraries \ & Pytorch, TensorFlow, Keras \\
Machine Learning \& Data Manipulation Libraries \ & Scikit-Learn, Pandas, NumPy \\
Visualization \ & Matplotlib, Seaborn\\
Operating Systems \ & Linux\\
Scientific Computing \& Containerization \ & OSG, Docker \\
Software \& Tools \ & Emacs, Jupyter Notebooks \\
Database Systems \ & MySQL, MongoDB \\
Web Technologies \ & Node, Express, React, Redux, REST\\
  Native \ & React Native

\end{tabular}

\end{rSection}

%----------------------------------------------------------------------------------------
%	WORK EXPERIENCE SECTION
%----------------------------------------------------------------------------------------
\begin{rSection}{Projects}
{\bf Research Projects}
\begin{cvEnum}
\item {\em Computer Vision:}
  \begin{itemize}
    \item (Ongoing) Develop and implement methods to
      identify pig posture. Achieved 98\% accuracy. (\textbf{TensorFlow})
     \item (Ongoing) Develop and implement deep methods for computational jigsaw
       puzzle solving. (\textbf{Test Driven Development, Python, OpenCV})

      \end{itemize}
    \item {\em Deep reinforcement learning:} Contributed to
      implementation of a reinforcement learning agent in different
      game environments.  Mined agent action data. (\textbf{TensorFlow,
        Keras, AWK})

    \item {\em Deep semantic hashing:} Develop and implement methods
      to find similarity preserving embeddings of data using
      locality-sensitive hashing. (\textbf{TensorFlow \& Keras})

    \item {\em Deep Generative Models for Optimization Problems}: (Ongoing) Develop and implement methods to solve optimization problems. (\textbf{PyTorch, TensorFlow})

    \item {\em Biological molecule target prediction:} Predicted
      binding interactions between biological molecules with Gaussian
      Mixture Models. (\textbf{Scikit-Learn \& Pandas}, super computing
      resources of \textbf{OSG}, and \textbf{Docker} for containerization)

\end{cvEnum}
\newpage
{\bf Side Projects}

{\em Machine Learning}
%    {\em Web technologies}
%    \begin{cvEnum}
%    \item Virtual drum set, Simon game
%    \end{cvEnum}
{\em Web \& Native Applications}
\begin{cvEnum}
  \item ToDo List
  \item Games: Pong, Simon, Game of Life, Matching Game (\textbf{React, Redux})
  \item Location-based job search app  (\textbf{React Native, Redux})

      \end{cvEnum}
\begin{cvEnum}
    \item Naive Bayes Classifier, Decision Tree Classifier (ID3)
    \item Document similarity using Locality Sensitive Hashing
    \end{cvEnum}


\end{rSection}



%	EXAMPLE SECTION
%----------------------------------------------------------------------------------------

\begin{rSection}{Relevant Courses}
{\em UNL}\\
Pattern Recognition (Deep Learning), Seminar in Deep Learning (3 Semesters)
\\Introduction to Machine Learning, Computational Intelligence (Neural Networks, Genetic Algorithms)
\\Algorithms for Large Scale Data
\\Statistical Methods in Research, Multivariate Statistics, Probability Theory\\
{\em MOOC (Coursera)}\\
Improving Deep Neural Networks, Structuring Machine Learning Projects
\end{rSection}




\begin{rSection}{Workshop \& Conference Presentations}

{\bf Workshop on Support Vector Machines,} {\em University of Nebraska Medical Center} \hfill{\em 2019}
\begin{cvEnum}
    Co-organized the workshop, contributed to hands-on session material preparation \& presentation.
\end{cvEnum}
{\bf Oral Presentation at International IEEE Conference BIBM,} {\em Kansas City} \hfill{\em 2017}
\end{rSection}
\begin{rSection}{Publications}

\end{rSection}
Dong Xu, Eleanor Quint, {\bf Zeynep Hakguder}, Haluk Dogan, Stephen Scott, and Matthew Dwyer. "Constraining Action Sequences with Formal Languages for Deep Reinforcement Learning." (2018).
\vspace*{0.05cm}

{\bf Zeynep Hakguder}, Jiang Shu, Chunxiao Liao, Kaiyue Pan, and Juan Cui. "Genome-scale MicroRNA target prediction through clustering with Dirichlet process mixture model." BMC genomics 19, no. 7 (2018): 658.
\vspace*{0.05cm}

{\bf Zeynep Hakguder}, Chunxiao Liao, Jiang Shu, and Juan Cui. "A new statistical model for genome-scale MicroRNA target prediction." In 2017 IEEE International Conference on Bioinformatics and Biomedicine (BIBM), pp. 101-107. IEEE, 2017.
\vspace*{0.05cm}

{\bf Zeynep M. Hakguder}, Dicle Yalcin, and Hasan H. Otu. "Bioinformatics approaches to single-cell analysis in developmental biology." MHR: Basic science of reproductive medicine 22, no. 3 (2015): 182-192.

\begin{rSection}{Academic Service}
 Subreviewer for IJCAI (International Joint Conferences on Artificial Intelligence) \hfill{2018 ---}
\end{rSection}
\begin{comment}
\begin{rSection}{References}
{\bf Stephen Scott,} {\em Advisor \& Instructor of Design and Analysis of Algorithms}
\\
{\bf Vinod Variyam,} {\em Advisor \& Instructor of Design and Analysis of Algorithms}
\\
{\bf Juan Cui,} {\em PI of SBBI \& Instructor of Introduction to Programming, Data Structures and Algorithms}
\\
{\bf Mohammad Hassan,} {\em Instructor of Introduction to Machine Learning}
\end{rSection}
\end{comment}
\end{document}

%%% Local Variables:
%%% mode: latex
%%% TeX-master: t
%%% End:
