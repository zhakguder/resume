%%%%%%%%%%%%%%%%%%%%%%%%%%%%%%%%%%%%%%%%%
% Medium Length Professional CV
% LaTeX Template
% Version 2.0 (8/5/13)
%
% This template has been downloaded from:
% http://www.LaTeXTemplates.com
%
% Original author:
% Rishi Shah
%
% Important note:
% This template requires the resume.cls file to be in the same directory as the
% .tex file. The resume.cls file provides the resume style used for structuring the
% document.
%
%%%%%%%%%%%%%%%%%%%%%%%%%%%%%%%%%%%%%%%%%

%----------------------------------------------------------------------------------------
%	PACKAGES AND OTHER DOCUMENT CONFIGURATIONS
%----------------------------------------------------------------------------------------

\documentclass{resume} % Use the custom resume.cls style

\usepackage[left=0.75in,top=0.6in,right=0.75in,bottom=0.6in]{geometry}
% Document margins
\usepackage{fontawesome}
\usepackage{comment}
\usepackage{hyperref}
\usepackage{enumerate}
\usepackage{comment}
\newcommand{\tab}[1]{\hspace{.2667\textwidth}\rlap{#1}}
\newcommand{\itab}[1]{\hspace{0em}\rlap{#1}}
\newenvironment{cvEnum}
{ \begin{enumerate}[]
    \setlength{\itemsep}{0pt}
    \setlength{\parskip}{0pt}
    \setlength{\parsep}{0pt}     }
{ \end{enumerate}                  }
\name{Zeynep Hakguder} % Your name
\address{\faHome \hspace{0.1em} 539 N 24th St Apt 19, Lincoln, NE 68503}
\address{\faMobile \hspace{0.1em}  +1 (531) 218-8301 \\ \faEnvelope
  \hspace{0.1em}
  \href{mailto:zphakguder@gmail.com}{zphakguder@gmail.com} \\ 
 \faGithub \hspace{0.1em}  \href{https://github.com/zhakguder}{https://github.com/zhakguder}}


\begin{document}

%----------------------------------------------------------------------------------------
%	EDUCATION SECTION
%----------------------------------------------------------------------------------------

\begin{rSection}{Education}

  {\bf University of Nebraska-Lincoln} \hfill { August 2017 - May
    2020} (Expected) 
\\ PhD in Computer Science

                        \small{
\begin{tabular}{cc}
{\bf Related Courses} &
                        
\begin{tabular}{lll}
  \textbf{{\em Deep Learning}} & \textbf{{\em Machine Learning}} & \textbf{{\em Probability}} \\
Pattern Recognition & Introduction to Machine Learning & Probability Theory\\
Seminar in Deep Learning & Computational Intelligence & Statistical Methods\\
     & Algorithms for Large Scale Data  & Multivariate Statistics\\
\end{tabular}
\end{tabular}
                        }
\end{rSection}

\begin{rSection}{Experience}
{\bf Teaching Assistant}, Head/Sole \hfill {\em Fall 2017 ---}
\begin{cvEnum}
\item Design and Analysis of Algorithms, Introduction to Machine
  Learning, Data Structures and Algorithms, Introduction to Python Programming  
\end{cvEnum}
{\bf Research Assistant} \hfill {\em January 2017-May 2018}
\begin{cvEnum}
\item SBBI Lab, Department of Computer Science and Engineering
    \end{cvEnum}

\end{rSection}


%----------------------------------------------------------------------------------------
%	WORK EXPERIENCE SECTION
%----------------------------------------------------------------------------------------
\begin{rSection}{Projects}
{\bf Research Projects}
\begin{cvEnum}
\item {\large{\em Computer Vision:}}
 Develop and implement deep methods for computational jigsaw
       puzzle solving. Decreased search space using approximate
       similarity search.\\
       \textbf{Python, OpenCV, Tensorflow, Docker,
       TDD}
     \vspace{0.3cm}

   \item {\large {\em Biological molecule target prediction:}} Predicted
     interactions between biological molecules with Gaussian
     Mixture Models using \textbf{Scikit-Learn \& Pandas}.Simulations
     were performed on a \textbf{distrib-uted
       computing} setting (OSG) and \textbf{Docker}. Built my own workflow
     using \textbf{Makefile, sed, AWK}.

\begin{comment}
    \item {\em Deep similarity search:} Develop and implement methods
      to find similarity preserving embeddings of data using
      locality-sensitive hashing. (\textbf{TensorFlow \& Keras})

    \item {\em Deep Generative Models for Optimization Problems}: (Ongoing) Develop and implement methods to solve optimization problems. (\textbf{PyTorch, TensorFlow})

\end{comment}
\end{cvEnum}
{\bf Side Projects}
\begin{cvEnum}
\item {\large{\em Location-based job search app}}  \\\textbf{React Native,
    Redux, Expo, REST, Firebase, Google Cloud Functions}
  \vspace{0.3cm}
\item {\large {\em ToDo List and games} (Pong, Simon, Game of Life, Matching
    Game)} \\
  \textbf{React, Redux, Node, Express, MongoDB, REST}

      \end{cvEnum}


\end{rSection}




%\begin{rSection}{Relevant Courses}
%{\em UNL}\\
%{\em MOOC (Coursera)}\\
%Improving Deep Neural Networks, Structuring Machine Learning Projects
%\end{rSection}


\begin{rSection}{Achievements, Service \& Leadership}
Head Teaching Assistant, {\em Design and Analysis of Algorithms}
\hfill {\em August 2018 ---}\\
3 peer-reviewed articles in journals and conferences, 1 preprint\\
Paper presentation @IEEE BIBM, {\em Kansas City, MO} \hfill{\em 2017}\\
Subreviewer, IJCAI \hfill {\em 2018 ---}\\
Workshop Co-organizer, \href
    {http://sbbi-panda.unl.edu/bibm2019/} {IRLB@IEEE BIBM,} {\em San
      Diego, CA} \hfill{\em Nov 18-21 2019}\\
Workshop Co-organizer, Support Vector Machines, UNMC, {\em Omaha, NE} \hfill{\em 2019}\\
Graduate student representative, Faculty Committee \hfill{\em{Aug
    2017-Aug 2018}}\\
Graduate student representative, Computer Science Program Committee
\hfill{\em{Aug 2019 ---}}\\

\end{rSection}

\end{document}

%%% Local Variables:
%%% mode: latex
%%% TeX-master: t
%%% End:
